\documentclass{article}
\usepackage[pdftex]{graphicx}
\usepackage{amsmath}
\usepackage{verbatim}
\usepackage{enumerate}
\author{Michael Anderson}
\title{Homework 1}
\begin{document}
\setlength{\parskip}{1em}
\maketitle
\center{ECE565}
\center{Prof. Raich}\\
\flushleft
\newpage
\thispagestyle{empty}
\mbox{}
\pagebreak

\section{}
\begin{enumerate}[(a)]
\item
$y(1) = b + e(1)$\\
$y(2) = a + e(2)$\\
$y(3) = b + e(3)$\\
$y(4) = a + e(4)$\\
\hspace{1pt} \vdots \\
$y(N) = \{a \hspace{4pt} or \hspace{4pt} b\} + e(N)$
\vspace{1em}

So the transform $H$ that describes this set of equations, and the parameter
list $\theta$, are given by:

\[
y(n) = H\theta + e(n) \rightarrow
\]

\[
y(n) =
\left[ \begin{array}{c c}
0 & 1 \\
1 & 0 \\
0 & 1 \\
1 & 0 \\
\vdots \\
\end{array}
\right]
\left[ \begin{array}{c}
a \\
b \\
\end{array} \right]
+ e(n)
\]

\vspace{15pt}

\item
\[
\hat{\theta}_{LS} = (H^TH)^{-1}H^Ty =
\]
\[
{\left(
\left[ \begin{array}{c c c c c}
0 & 1 & 0 & 1 & \\
1 & 0 & 1 & 0 & \ldots \\
\end{array} \right]
\left[ \begin{array}{c c}
0 & 1 \\
1 & 0 \\
0 & 1 \\
1 & 0 \\
\vdots \\
\end{array}
\right]
\right)}^{-1}
\left[ \begin{array}{c c c c c}
0 & 1 & 0 & 1 & \\
1 & 0 & 1 & 0 & \ldots \\
\end{array}
\right]
\left[ \begin{array}{c}
y(1)\\
y(2)\\
y(3)\\
y(4)\\
\vdots\\
\end{array} \right]
=
\]

\[
{\left[ \begin{array}{c c}
\lfloor N/2 \rfloor & 0 \\
0 & \lceil N/2 \rceil \\
\end{array} \right]}^{-1}
\left[ \begin{array}{c}
y(2) + y(4) + y(6) + \cdots \\
y(1) + y(3) + y(5) + \cdots\\
\end{array} \right]
=
\]

\[
\left[ \begin{array}{c}
\frac{y(2) + y(4) + y(6) + \cdots}{\lfloor n/2 \rfloor} \\
\\
\frac{y(1) + y(3) + y(5) + \cdots}{\lceil n/2 \rceil} \\
\end{array} \right]
\]
\vspace{6pt}
\item
With the change of parameters we now have:
\[
y(n) = HM\theta + e(n)
\]

To return to the canonical form, let $\theta' = M\theta$, so
$\theta = M^{-1}\theta'$. Absorb the $M^{-1}$ term into $H'$ to get 
$H' = HM^{-1}$. So to estimate in terms of $\theta'$:

\[
\hat{\theta}'_{LS} = (H'^TH')^{-1}H'^Ty = ((HM^{-1})^{T} HM^{-1})^{-1}(HM)^{-1T}y =
\]

\[
(M^{-1T} (H^T H) M^{-1})^{-1}M^{-1T}H^Ty = 
M (H^T H)^{-1} M^T M^{-1T} H^T y =
\]

\[
M (H^T H)^{-1} H^T y = M \hat{\theta}_{LS}
\]

\vspace{2em}

\item
$c = a/2 + b/2$ and $d = a/2 - b/2$ written as a linear transform $M$ is:

\[
M = \left[ \begin{array}{c c}
1/2 & 1/2\\
1/2 & -1/2\\
\end{array} \right]
\]

Now:

\[
\hat{\theta}'_{LS} = M\hat{\theta}_{LS} = 
\left[ \begin{array}{c c}
1/2 & 1/2\\
1/2 & -1/2\\
\end{array} \right]
\left[ \begin{array}{c}
\frac{y(2) + y(4) + y(6) + \cdots}{\lfloor N/2 \rfloor} \\
\\
\frac{y(1) + y(3) + y(5) + \cdots}{\lceil N/2 \rceil} \\
\end{array} \right] =
\left[ \begin{array}{c c}
\frac{\sum_{i=1}^{N} y(i)}{N}\\
\\
\frac{(y(2) + y(4) + y(6) + \cdots) - (y(1) + y(3) + y(5) + \cdots)}{N}\\
\end{array} \right]
\]

The result fits with intuition. Our $c$ is estimated by the average of all
observations, and $d$ is estimated by the average difference of the even 
($a$ valued) and odd ($b$ valued) observations.

\end{enumerate}

\section{}
\begin{enumerate}[(a)]
\item
$y(n-N+1) = A + \lambda^{-N+1} e(n-N+1)$\\
$y(n-N+2) = A + \lambda^{-N+2} e(n-N+2)$\\
$y(n-N+3) = A + \lambda^{-N+3} e(n-N+3)$\\
\hspace{1pt} \vdots \\
$y(n) = A + \lambda^0 e(n) = A + e(n)$
\vspace{1em}

The problem states that the $y$ values in question are between $y(n-N+1)$ and 
$y(n)$ inclusive, $A$ appears alone in the $H\theta$ term of each equation as
the only parameter, and the error weights are simply the diagonals of $W$, so
the setup is:

\[
y(n) = H\theta + We(n) \rightarrow
\]

\[
\left[ \begin{array}{c}
y(n-N+1)\\
y(n-N+2)\\
y(n-N+3)\\
\vdots\\
y(n)
\end{array} \right] =
\left[ \begin{array}{c}
1\\
1\\
1\\
\vdots\\
1\\
\end{array}\right]
[A] +
\left[ \begin{array}{c c c c c}
\lambda^{-N+1} & 0 & 0 & & 0 \\
0 & \lambda^{-N+2} & 0 & & 0 \\
0 & 0 & \lambda^{-N+3} & & 0\\
& & & \ddots & \vdots \\
0 & 0 & 0 & \cdots & 1\\
\end{array} \right] 
\left[ \begin{array}{c}
e(n-N+1)\\
e(n-N+2)\\
e(n-N+3)\\
\vdots\\
e(n)
\end{array} \right]
\]

\vspace{15pt}

\item

\[
\hat\theta_{WLS} = (H^T W H)^{-1} H^T W y =
\]

\[
\left( [\begin{array}{c c c c c}
1 & 1 & 1 & \cdots & 1
\end{array}]
\left[ \begin{array}{c c c c c}
\lambda^{-N+1} & 0 & 0 & & 0 \\
0 & \lambda^{-N+2} & 0 & & 0 \\
0 & 0 & \lambda^{-N+3} & & 0\\
& & & \ddots & \vdots \\
0 & 0 & 0 & \cdots & 1\\
\end{array} \right]
\left[ \begin{array}{c}
1\\
1\\
1\\
\vdots\\
1\\
\end{array}\right] \right)^{-1}
\]

\[
[\begin{array}{c c c c c}
1 & 1 & 1 & \cdots & 1
\end{array}]
\left[ \begin{array}{c c c c c}
\lambda^{-N+1} & 0 & 0 & & 0 \\
0 & \lambda^{-N+2} & 0 & & 0 \\
0 & 0 & \lambda^{-N+3} & & 0\\
& & & \ddots & \vdots \\
0 & 0 & 0 & \cdots & 1\\
\end{array} \right]
\left[ \begin{array}{c}
y(n-N+1)\\
y(n-N+2)\\
y(n-N+3)\\
\vdots\\
y(n)\\
\end{array}\right] =
\]

\[
\left( \frac{1}{\sum_{i=0}^{N-1} \lambda^{-i}} \right)
\left( \sum_{i=n-N+1}^n \lambda^{i-n}y(i) \right)
\]

\vspace{6pt}

Since $\lambda^{-i} = (\frac{1}{\lambda})^i$, and recalling that
$\sum_{i=0}^{\infty} r^i = \frac{1}{1-r}$, and letting $N$ go to infinity,
yields:

\[
\frac{1}{\frac{1}{1-\frac{1}{\lambda}}}\left( \sum_{i=-\infty}^n
\lambda^{i-n}y(i) \right) = \left( 1-\frac{1}{\lambda} \right)
\left( \sum_{i=-\infty}^n \lambda^{i-n}y(i) \right)
\]

\end{enumerate}
\end{document}
